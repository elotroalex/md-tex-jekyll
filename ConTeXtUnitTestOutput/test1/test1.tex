\setvariables[article][shortauthor={Fanon,Ballsun-Stanton,Gil}, date={Febuary 2016}, issue={102}, DOI={10.1215/07990537-2739812}]

\environment env_journal

\startcomponent poetry


\startchapter[title={A long title for the first unittest: exploring some features and
questions of the nature of unit tests, poetry, and red lines.}, marking={Poetry and Redlines}, reference={poetry}]


\startlines
{\bf
Franz Fanon
Brian Ballsun-Stanton
Alex Gil
}
\stoplines

\subsection[french]{French!}

Il est désormais difficile de nier\footnote{A rare first page footnote}
le changement\footnote{oooh} fondamental\footnote{just trying to} de
paradigme\footnote{argh} que représente l'édition numérique des textes.
Du projet Gutenberg de 1971 au projet HyperNietzche actuellement
développé au sein de l'ITEM, les éditions numériques ont évolué d'une
simple transcription de texte jusqu'à devenir des outils perfectionnés
d'analyse, qui ont changé notre façon de comprendre et d'étudier les
textes. Tout au long de ce processus, malheureusement, de nombreux
corpus importants de la littérature ont été laissés pour compte. Au
cours des années 1990, la généralisation de l'internet a coïncidé avec
une migration en masse des textes papier vers un support numérique. Mais
cet effort étant nécessairement lié à un appui institutionnel et
financier massif, la plus grande partie de ces efforts de numérisation
ont porté sur les canons bibliographiques de l'Amérique du Nord et de
l'Europe. D'autres champs littéraires, dont l'immense majorité des
corpus de la sphère francophone (Afrique, Caraïbe essentiellement),
n'ont pas été pris en compte, pour des raisons diverses
(institutionnelles, technologiques, juridiques etc.) En ce début de
XXIème siècle, au moment où l'attention s'est déplacée de la
numérisation à la création d'outils d'analyse qui pourraient être
appliqués à l'énorme quantité de données collectées, il est devenu
encore plus difficile d'obtenir des fonds pour les projets de
numérisation à grande échelle de ces littératures négligées. Une
solution provisoire, donc, consiste à créer des éditions numériques
d'œuvres-clé de ces grands corpus, en laissant de côté pour le moment la
numérisation massive des fonds d'archives. L'édition d'une pièce d'Aimé
Césaire se propose donc un but démonstratif et exemplaire. Appelons
cette solution la micro-numérisation. n'ont pas été pris en compte, pour
des raisons diverses (institutionnelles, technologiques, juridiques
etc.) En ce début de XXIème siècle, au moment où l'attention s'est
déplacée de la numérisation à la création d'outils d'analyse qui
pourraient être appliqués à l'énorme quantité de données collectées, il
est devenu encore plus difficile d'obtenir des fonds pour les projets de
numérisation à grande échelle de ces littératures négligées. Une
solution provisoire, donc, consiste à créer des éditions numériques
d'œuvres-clé de ces grands corpus, en laissant de côté pour le moment la
numérisation massive des fonds d'archives. L'édition d'une pièce d'Aimé
Césaire se propose donc un but démonstratif et exemplaire. Appelons
cette solution la micro-numérisation.

\section[a-list]{A list}

\startitemize[packed]
\item
  normal list
\item
  this should have bullets
\stopitemize

\section[some-poetry-with-two-stanzas]{Some poetry with two stanzas}

\startlines\setupindenting[no]

rather bear those ills we had,
\strut \color[red]{Than fly to others, that we knew not of.}
\strut ~~~~But brian can't write poetry
\strut \color[red]{~~~~But he can indent}
\strut blockquoted lines
\strut \color[red]{This is}
\strut a test
\strut ~~~~to compel some spaces
\strut \color[red]{~~~~though this is at the same ul level.}

and this is back
\strut because poets.
\strut base line
\strut ~base line indent 1
\strut ~~base line indent 2
\strut ~~~base line indent 3
\strut ~~~~base line indent 4
\strut ~~~~~indented line
\strut ~~~~~~~~~8 spaces

\stoplines

\section[some-poetry-with-no-stanzas]{Some poetry with no stanzas}

\startlines\setupindenting[no]

rather bear those ills we had,
\strut Than fly to others, that we knew not of.
\strut ~~~~But brian can't write poetry
\strut ~~~~But he can indent
\strut blockquoted lines
\strut This is
\strut a test
\strut ~~~~to compel some spaces
\strut ~~~~though this is at the same ul level.

\stoplines

\startitemize[packed]
\item
  but above
\item
  should not have bullets
\stopitemize

\section[a-red-header-thingo]{\color[red]{A red header thingo}}

\subsection[a-sub-red-header-thingo]{\color[red]{A sub red header
thingo}}

\subsubsection[a-level-threeeeeee-header-thingo]{\color[red]{A level
threeeeeee header thingo}}

Parmi les nombreux défis qui sont à relever par les équipes d'édition
numérique, dans le cas d'objets éditoriaux complexes, l'un des plus
délicats reste de réussir la collecte et de bien choisir ses stratégies
de représentation éditoriale. Les textes de Césaire publiés sous le
titre Et les chiens se taisaient sont l'un de ces objets éditoriaux
complexes. Réunir dans un seul espace en ligne les matériaux disparates
(tapuscrits, livres imprimés, enregistrements, film) qui constituent
cette archive sera déjà une réussite considérable. Mais nous désirons ne
pas nous arrêter là. Si la tâche de l'édition génétique, comme Walter
Benjamin le disait de l'histoire matérialiste, consiste à « prendre
possession de la mémoire, dans le cillement d'un éclair au moment du
danger »\footnote{This is a footnote?} -- en l'occurrence une suite
d'actes poétiques réalisés par Aimé Césaire et ses
collaborateurs\footnote{another footnote}-- alors il nous faut une
édition qui puisse : 1) isoler ce que Jerome McGann\footnote{Shockingly,
  a footnote} appelle le code linguistique (les mots) et le code
bibliographique (l'arrangement, la typographie, le support matériel,
etc. de façon à ne pas déstabiliser leur équilibre délicat\footnote{The
  footnotes come marching, one by one\ldots{}} ; 2) représenter de façon
dynamique la trajectoire et l'interrelation entre les éléments des
différents états du texte, afin de nous permettre de recréer le long
processus génétique de l'archive. L'absence de coordination entre ces
deux fonctionnalités a imposé jusqu'à aujourd'hui une séparation entre
les outils d'analyse et la représentation des textes en ligne. Pour
nous, l'avenir de l'édition génétique suppose la capacité de joindre
l'analyse à la représentation. Les ressources technologiques n'ont pas
encore permis de concrétiser toutes nos intentions, mais notre but est
de faire reculer les limites actuelles.

\startblockquote
At the limit of an always increasing elimination of references and
finalities, an ever-increasing loss of resemblances and designations, we
find the digital and programmatic sign, whose \quotation{value} is
purely {\em tactical}, at the intersection of other signals
(\quotation{bits} of information/tests) whose structure is that of a
micromolecular code of command and control.\footnote{Jean Baudrillard
  and Mark Poster, {\em Selected Writings} (Stanford, Calif.: Stanford
  University Press, 1988), 139--40.}
\stopblockquote

Roman Jakobson called such construction and deconstruction of meaning
the \quotation{profuse exchange of ritualized formulas} or the phatic
function of language.\footnote{Bronislaw Malinowski et al., “The Problem
  of Meaning in Primitive Languages,” in {\em The Meaning of Meaning; a
  Study of the Influence of Language Upon Thought and of the Science of
  Symbolism,} ed. C. K Ogden (London; New York: K. Paul, Trench, Trubner
  & Co.; Harcourt, Brace & Co., 1923), 146; Roman Jakobson, “Closing
  Statement: Linguistics and Poetics,” in {\em Style in Language}, ed.
  Thomas A Sebeok ({[}Cambridge: Technology Press of Massachusetts
  Institute of Technology, 1960), 355.} In the phatic function lies the
essence of programming. Code shapes and commands. At the same time, it
conjures fantastical metaphors to occlude the structure of shaping and
commanding. Simulation obscures the incongruence between visible
representation and the underlying material affordances of the medium.
What you see is not always what you get. We are instead confronted with
a composite image, which under examination reveals a complex process of
transfiguration between the visible sign and the sign at the site of the
inscription. When reading online, for example, we observe what looks
like a book, where we should also perceive an attempt to sensor and
surveil. The simulation is without a referent. It bares no resemblance
to the material substratum of electronic reading. We believe we are
handling a book. Our ideas about reading and interpretation subsequently
rely on that initial physical point of contact with paper. But when
reading electronically, we are handling something other than print
material. The resemblance to paper guides our intuitions about the
possibilities of the medium: the ability to scroll, bookmark, or turn
pages. We have far fewer intuitions about the affordances of inscription
at the micromolecular level. As we \quotation{turn the page} an electric
charge crosses the impenetrable oxide barrier to reach the floating gate
through quantum tunnelling.\footnote{P. Pavan et al., “Flash Memory
  Cells-an Overview,” {\em Proceedings of the IEEE} 85, no. 8 (August
  1997): 1248--71,
  doi:\useURL[url1][https://doi.org/10.1109/5.622505][][10.1109/5.622505]\from[url1];
  V.-Y. Aaron and J. Leburton, “Flash Memory: Towards
  Single-Electronics,” {\em IEEE Potentials} 21, no. 4 (October 2002):
  35--41,
  doi:\useURL[url2][https://doi.org/10.1109/MP.2002.1044216][][10.1109/MP.2002.1044216]\from[url2];
  R. Bez et al., “Introduction to Flash Memory,” {\em Proceedings of the
  IEEE} 91, no. 4 (April 2003): 489--502,
  doi:\useURL[url3][https://doi.org/10.1109/JPROC.2003.811702][][10.1109/JPROC.2003.811702]\from[url3].}

\section[a-black-header-thingo]{A black header thingo}

\subsection[a-sub-balck-header-thingo]{A sub balck header thingo}

\subsubsection[a-level-threeeeeee-header-thingo-1]{A level threeeeeee
header thingo}

We should talk about this template. Some things to discuss:

\startitemize[packed]
\item
  separate bib file?
\item
  yaml metadata
\item
  tests for correct submission
\stopitemize

Unlike figurative language, machine control languages function in the
imperative. They do not stand for action---they are action. Code
represents only the exercise of power. More binding than the
\quotation{speech acts} J.L. Austin, control codes arrange and regulate.
The difference between representation and control is one of brute force.
It lies in the distinction between a restraining order and physical
restraint. A restraining order {\em represents} the calling forth of
codified power. Physical restraints like handcuffs {\em enact} the
exercise of codified power. Like all violence they do not stand for
anything. Stripped of references, resemblances, and designations, they
are in themselves an arrangement and rearrangement of matter. The
handcuffs contort the body into the shape of submission. Absent a body,
the restraints draw an empty shape.

Code acts similarly to shape the word. Located somewhere between idea
and material, formatting relates content to matter. It mediates by
imposing structure. Think of a paragraph, for example. By convention,
writers use paragraphs to break up the flow of monolithic thought on a
page. The paragraph contains information. Can we imagine an empty
paragraph? Can the shape of the paragraph persist outside of the
material confines of the page or the screen? Can one imagine a paragraph
that unfolds spatially not in two dimensions, but in one, along a
straight line? What about a three dimensional paragraph? Could it take
the shape of a cuboid instead instead of a rectangle? These questions
boggle the mind because the paragraph draws a singular shape. It is a
textual container of a type. Any other shape less or more than the
paragraph would go by another name. It would constitute another format.
To imagine something like a one-dimensional paragraph is akin to
imagining a flat shoebox. A flat shoebox could no longer hold shoes. It
would contain something else like images of footwear. Similarly, a
paragraph identifies a particular arrangement of elements. It is a box
or a data structure of a shape, made to hold words and sentences. Like
nesting dolls, words and sentences are in themselves data structures
that contain further, smaller arrangements of information. One could
say, what of such arrangements? Who cares about paragraphs? It is merely
one type of a container among many. It has only an instrumental function
to help get the point across. The meat of interpretation lies in the
stuff within. Words come in other shapes and sizes. The outer container
is disposable and therefore insignificant.

Formats could only seem insignificant in the past when they were few and
simple. The transition between static and dynamic media necessitates
renewed attention to the formatting layer of meaning making. What you
saw is what you got on the page. On the screen, what you see is but a
small part of what you get. The content---all that is contained on a
page---shifts beneath the projected image. In print, content can be
gleaned from the surface. There is nothing but surface on a page. The
screen is a layered surface. Sandwiched between panes of glass, liquid
crystal moves in response to electrical modulation. The ebb and flow of
electricity in turn reflects yet another layer of codification,
inscribed onto yet other recondite planes of inscription. A byte, made
up of eight binary bits, holds a letter. The string of letters spelling
out \quotation{hello world} occupies eleven bytes, if you count the
space. A file in the Portable Document Format (\type{.pdf}) containing
nothing but \quotation{hello world} takes up 24,335 bytes on my system.
Encoding accounts for the disparity between plain text and fancy text,
the latter defined as \quotation{text representation consisting of plain
text plus added information.}\footnote{Unicode Consortium, {\em The
  Unicode Standard: Worldwide Character Encoding, Version 1}
  (Addison-Wesley, 1990).}

\placefigure{Forms and formats.}{\externalfigure[images/forms.png]}

In cognitive metaphor theory, the kind of resemblances that we have been
discussing so far are called {\em structural metaphors}.
\quotation{Structural metaphors allow us to do much more than just
orient concepts,} Lakoff and Johnson wrote.\footnote{George Lakoff and
  Mark Johnson, {\em Metaphors We Live by} (Chicago: University of
  Chicago Press, 1980), 61.} Grounded in \quotation{systematic
correlations within our experience,} structural metaphors transfer
organizing principles from one domain to another. Thus to say something
like \quotation{time is money} is to suggest that something in the
arrangement of the financial system correlates systematically to
something in the arrangement of the temporal system.\footnote{Ibid.,
  65--68.} It is a structural and not just a semantic similarity. If
metaphors work by transferring qualities, structure is the quality being
transferred in all of the above cases. Structural metaphors organize one
thing in the shape of another. They are for this reason key to
understanding the transference that takes place in the interface between
human and machine.

\startblockquote
a shared books with database metaphor, a reference books metaphor, and a
card catalog metaphor in one system that allows large object oriented
data bases to be organized and accessed in an exclusive environment and
in addition allows access to screen icons, creates a visual hierarchy of
related and shared objects, and allows mutually exclusive access to the
metaphors within the library.\footnote{Henry G. Pajak, “Electronic
  Library,” February 1992, 1,
  \useURL[url4][http://www.google.com/patents/EP0472070A2]\from[url4].}
\stopblockquote

In the so-called classical view, metaphors are simply a type of
figurative language. To say \quotation{the day stands tiptoe on the
misty mountain tops} is to use the verb \quotation{stand} in a novel
linguistic context. Days have no literal legs to stand on. John Searle,
George Lakoff, and Mark Turner, among others, have argued that metaphors
are more broadly a cognitive phenomenon, mapping distinct ontological
categories across \quotation{conceptual domains.}\footnote{Lakoff and
  Johnson, {\em Metaphors We Live by}; Mark Turner, {\em Death Is the
  Mother of Beauty: Mind, Metaphor, Criticism} (Chicago: University of
  Chicago Press, 1987); George Lakoff, “The Contemporary Theory of
  Metaphor,” in {\em Metaphor and Thought}, ed. Andrew Ortony
  (Cambridge: Cambridge University Press, 1998), 201--52, John R.
  Searle, “Metaphor,” in {\em Metaphor and Thought}, ed. Andrew Ortony
  (Cambridge: Cambridge University Press, 1998), 83--112.} In the
modified view, even such basic semantic concepts as \quotation{state,
quantity, action, cause, purpose, means, modality} are metaphorical in
nature.\footnote{George Lakoff and Mark Johnson, “The Metaphorical
  Structure of the Human Conceptual System,” {\em Cognitive Science} 4,
  no. 2 (April 1980): 195--208,
  doi:\useURL[url5][https://doi.org/10.1016/S0364-0213(80)80017-6][][10.1016/S0364-0213(80)80017-6]\from[url5];
  Lakoff, “The Contemporary Theory of Metaphor,” 212.} Beyond figurative
lyrical language, Lakoff and others have argued that metaphors broadly
structure everyday experience. Thus the analysis of common phrases like
\quotation{things are looking up} and \quotation{I can't get that tune
out of my mind} reveals underlying figuration like \quotation{good
things are up} and \quotation{the mind is a container.} The idea that
\quotation{good things are up} generates a multitude of metaphors like
\quotation{profits are going up} and \quotation{moving on up,} for
example\footnote{Lakoff and Johnson, “The Metaphorical Structure of the
  Human Conceptual System,” 195--98.}

\placefigure{\quotation{An exemplary interface for viewing a three
dimensional book.}\footnote{Stuart Kent Card et al., “Methods, Systems,
  and Computer Program Products for the Display and Operation of Virtual
  Three-Dimensional Books,” March 2006, 3,
  \useURL[url6][http://www.google.com/patents/US7015910]\from[url6].}}{\externalfigure[images/book-metaphor.png]}

In the cognitive view, the metaphor performs a number of
\quotation{conventional mappings from one domain to another.}\footnote{Lakoff,
  “The Contemporary Theory of Metaphor,” 239.} Lakoff mentions for
example the common trope of \quotation{a state is a person,} implicit in
the ideas of \quotation{friendly} and \quotation{hostile}
states.\footnote{Ibid., 243.} This metaphor implies that ideas about
agency, emotion, and mental life usually attached to people can be
extended to state actors. Similarly, to say that someone is
\quotation{boiling mad,} instantiates the common trope of
\quotation{anger is a hot liquid in a container.} In this case, common
known properties attached to the domain of physics are mapped onto the
domain of emotion. Lakoff further explains that such domain mappings
tend to follow a few rules. They are usually partial and asymmetrical.
\quotation{Mappings are not arbitrary,} he writes, \quotation{but
grounded in the body and in everyday experience and knowledge.} Finally
domain mappings obey what Lakoff calls the Invariance Principle, by
which \quotation{the image schema structure of the source domain is
projected onto the target domain in a way that is consistent with
inherent target domain structure.}\footnote{ibid. See also Mark Turner
  and Gilles Fauconnier, “Conceptual Integration and Formal Expression,”
  {\em Metaphor and Symbolic Activity} 10, no. 3 (September 1995):
  183--204,
  doi:\useURL[url7][https://doi.org/10.1207/s15327868ms1003_3][][10.1207/s15327868ms1003_3]\from[url7];
  Francisco José Ruiz de Mendoza Ibáñez, “On the Nature of Blending as a
  Cognitive Phenomenon,” {\em Journal of Pragmatics} 30, no. 3
  (September 1998): 259--74,
  doi:\useURL[url8][https://doi.org/10.1016/S0378-2166(98)00006-X][][10.1016/S0378-2166(98)00006-X]\from[url8];
  George Lakoff, “The Invariance Hypothesis: Is Abstract Reason Based on
  Image-Schemas?” {\em Cognitive Linguistics} 1, no. 1 (2009): 39--74,
  doi:\useURL[url9][https://doi.org/10.1515/cogl.1990.1.1.39][][10.1515/cogl.1990.1.1.39]\from[url9].}

\subsubsection[another-section]{Another Section}

The principles of metaphor-driven design contain an implicit model of
human--computer interaction, which implies that humans prefer to
manipulate digital information stored on computational media by the
means of familiar mediating structures---paragraphs, pages, files, and
folders---associated figuratively with the affordances of print media.
We know, in other words, what paragraphs, pages, files, and folders can
do on paper and we would like for digital images of paper to behave in a
similar way. For example, one affordance of paper is that it can be
folded. It therefore becomes possible to \quotation{earmark} a page by
folding a corner. The fold enables subsequent recollection of text that
has been previously read. By these means, a reader marks a notable place
in the text in order to return to it later. The digital medium cannot be
folded in the same way as it offers a set of physical affordances that
differ from actual paper. Readers are not familiar with \quotation{what
can be done} digitally, however; consequently, the affordances of
digital media are presented through metaphor. Thus a virtual
\quotation{earmark} on a \quotation{page} represents a numerical pointer
to a specific address in the computer memory. A \quotation{page} stands
for a range of related addresses that correspond roughly to the
information visible on an analogous page in print. Similarly one
\quotation{drops a folder into a trash bin} or \quotation{drags and
drops a file} or \quotation{bookmarks a page} on a screen. Such
metaphors rely on habituated insight with one medium extended into
another. We do not literally \quotation{drag} or \quotation{drop} bits,
but we use metaphors of paper and trashcan to help us manipulate bits
and bytes as if they were household objects. The metaphor opens
figurative possibilities, but it also obscures the actual physical
contingencies of interacting with bits and bytes, logic gates and
magnetic traces.

The affordances of the physical medium differ from those of the
simulated one. We manipulate bits and bytes differently from files and
folders, pages and paragraphs. The metaphorical substitution encourages
readers to extend the facility they have with manipulating one sort of
media (paper and ink) to another (screen and pixels). What readers gain
in facility, they lose in critical faculty. Alienated from the actual
physical structures of information storage and retrieval, readers gain
access to the metaphor alone. Thus we go through the motions of turning
the page but actually redraw the screen. We \quotation{highlight a
passage,} an action that may also send information about the highlighted
passage to a data aggregation service. We \quotation{share a book} which
means really assigning a temporary license to another user. The
structures of governance do not reveal themselves in the metaphor. Where
did the text go? someone asks when downloading a paper from an online
journal.\footnote{The notion of \quotation{digital text} itself is a
  metaphor. Files do not really hold texts. The idea of \quotation{text}
  identifies a segment of stored memory coupled with control codes that
  govern layout and projection in specific material context. Together,
  these diverse signals and physical affordances create the illusion of
  a single text.} It is in your \quotation{home} I answer. But unless
one of us is familiar with the material contingencies of file storage,
neither has a mental map of any physical location corresponding to the
\quotation{home} directory, the default location of personal files on
many systems. When confronted with the actual affordances of digital
text, the user grasps for neutered metaphors. We \quotation{reside} in
such homes, \quotation{own,} \quotation{share,} and \quotation{create}
only in the simulacrum.

Metaphors of human-computer interaction conceal the structure of
computation. Print offers a relatively static and stable medium for
knowledge transmission. Ink and paper do not change in transit. By
contrast, the vessels of computation are capable of altering the content
dynamically. Imagine me asking you to read Shakespeare's {\em Hamlet},
for example, by lending you a copy of the text. In the case of a paper
book, I may be sure that the text in my hands will remain the same as I
pass it into yours. But the computed sign also has the capability to
adjust itself to new contexts. For example, the simulated {\em Hamlet}
may adapt to the new reader's geographic location, mood, or consumption
habits. In fact, most texts we consume today come to us in such
computationally constructed way. The front page of the New York Times
viewed in Beijing will differ from the front page viewed in New York.
The two \quotation{pages} or \quotation{sites} are in some sense two
completely different texts. But in another sense, the \quotation{front
page} identifies the same location of the same text, in two diverging
and dynamically composed versions. They feed off of the same sources.
The same source code gives rise to both texts.

The key to understanding \quotation{the loss of resemblances} that
accompanies ubiquitous simulation lies in the inner dynamics of metaphor
machinery. A functioning metaphor, if you would recall from Lakoff, is
one which ferries the schematic composition of one domain into another.
Thus to say \quotation{life is a stage} is to transpose something about
theater onto life. In literary terms, the theater is \quotation{tenor}
where \quotation{life} is \quotation{vehicle} of the composite
figure.\footnote{I. A Richards, {\em The Philosophy of Rhetoric} (New
  York; London: Oxford University Press, 1936).} Simulations work
differently. Where the tenor of the literary metaphor crosses several
semantic and cognitive domains, the computational metaphor substitutes
the \quotation{signs of the real for the real.}\footnote{Jean
  Baudrillard, {\em Simulacra and Simulation} (Ann Arbor: University of
  Michigan Press, 1994), 2.} It is a subtle difference that engenders
not-so-subtle effects. For example, it is one thing to say
\quotation{you are the apple of my eye} and quite another to actually
confuse apples for eye pupils. Baudrillard gives us the example of a map
that no longer corresponds to any territory. He calls such a condition
of pure simulation without a referent {\em hyper-reality}. We expect a
digital simulation to attain a measure of correspondence between
representation and the thing being represented. For example, in theory,
a weather simulation should be capable of modeling observed
meteorological conditions. But would it be a weather simulation if the
model was broken in some way, or, in the extreme, if it had no
correspondence to the physics of clouds, wind, and water? The hyper-real
breaks further still by usurping the underlying reality. The model does
not merely obscure, it takes place of the thing being modeled. In other
words, it begins to simulate itself, according to its own rules, similar
only to itself. The simulation no longer corresponds to any situation
\quotation{on the ground.} Severed from its referent, the symbol itself
attains the status of reality. Thus hyper-reality: a symbol that folds
onto itself. It is a weather simulation confused for weather.

We know that physical affordances of liquid crystal displays (LCDs) and
magnetic storage differ drastically from those of paper, goat skins, or
parchment. Yet digital surface representation maintains the illusion of
self-similarity. We are faced with what is called {\em skeuomorphic}
design, by which screen reading resembles print. In this way, an
electronic book reader simulates the bent corner of a well-thumbed book.
The skeuomorphic resemblance itself constitutes a metaphor worthy of
critical examination. The principles of skeuomorphic design extend a
visual metaphor from one medium to another. The reader already knows how
to turn pages of a book. A book device therefore simulates pages to ease
the cognitive burden of transitioning from paper to pixel. Instead of
issuing unfamiliar commands to the computer to turn the page, readers
perform the more habituated motion of swiping across the screen. The
gliding motion enacts a kinetic analogy---a type of a
metaphor---transposing properties of paper to glass.

The interface metaphor similarly exchanges one referent for another.
Simulation should, by definition, \quotation{assume a form resembling
that of something else.}\footnote{“Simulation, N.2,” {\em OED Online}
  (Oxford University Press, December 2015),
  \useURL[url10][http://www.oed.com/view/Entry/100790]\from[url10].}
Metaphor machines assume the form of one thing, while structuring
another. To drag and drop a document into a trashcan on the screen, for
example, should in theory correspond to an analogous set of data
manipulations on the disk. Yet, \quotation{discarding a file} in this
manner does not necessarily include deletion of data from the storage
medium, as expected. The representation of the document may disappear
visually where the inscription endures. Such \quotation{loss of
resemblances} could be insignificant. Does one care whether a file was
actually erased or not when performing deletion? Perhaps not in many
cases. But in some cases, when it really matters---under the threat of
censorship or prosecution, for example---the incongruence exposes the
frailty of our alienation from the material contexts of digital
knowledge production. We want the thing to stay deeply deleted. Our
grasp on the medium weakens the more convincing the simulacrum.

Readers bear the burden of conceptual transference. In pretending to
turn virtual pages, we lose sight of the mechanisms producing the
simulation. If we hope to practice anything like interpretation or close
reading at the level of discourse, we must certainly also practice them
at the physical site of discourse formation. A truly materialist poetics
would reach beyond representation towards the object. More than
superficial embellishment, the skeuomorphic metaphor enacted at the
surface of the digital literary device structures all meaning-carrying
units: from individual letters, to words, paragraphs, chapters, pages,
and books. We know that there is nothing inherently page-like about
rigid slabs of glass and silicone. The metaphor of \quotation{turning
the page by swiping across the screen} conceals the structural rift
between media.

Why would readers engage in such a charade? Why not simply make use of
novel interfaces afforded by new technology? The literature from the
field of human--computer interaction suggests a formalist answer:
habituation.\footnote{J.M. Carroll and John C. Thomas, “Metaphor and the
  Cognitive Representation of Computing Systems,” {\em IEEE Transactions
  on Systems, Man and Cybernetics} 12, no. 2 (March 1982): 107--16,
  doi:\useURL[url11][https://doi.org/10.1109/TSMC.1982.4308795][][10.1109/TSMC.1982.4308795]\from[url11];
  J. M Carroll, R. L Mack, and W. A Kellogg, {\em Interface Metaphors
  and User Interface Design} (San Jose {[}etc.{]}: IBM Thomas J. Watson
  Research Division, 1987); Joel Spolsky, {\em User Interface Design for
  Programmers} (Berkeley, CA; New York, NY: Apress, 2001).} The initial
effort it takes to learn to read in a new environment may discourage
potential readers from adopting a new technology. Smart designers
therefore rely on acculturated practice, the turning of pages in our
case, to minimize the \quotation{friction} of adoption. Although an
\quotation{electronic book reader} contains no pages as such, it extends
the metaphor of pages to electronic reading. The perceived facility of
use comes at a cost of critical engagement. The structures of
governance, like those embedded in a Portable Document Format, remain
for the most part inaccessible to analysis. A digital poem, a novel, a
physician's script, or a legal contract resembles their paper
counterparts to enable familiar actions. But while imitating paper
pages, the reading {\em appliance} also monitors, adjusts, warns, and
controls.

Simulation conceals structuring principles large and small. Some of the
concealed details may remain inconsequential, like the limit on how many
keys can be pressed at once without overwhelming the circuitry of
keyboard when typing. Other concealed details are of paramount
importance, like digital rights management chips and censorship filters.
Electronic books are governed and internalize governing structures in
ways that are often purposefully hidden from the reader. For example,
the US Digital Millennium Copyright Act prohibits the physical
circumvention of copyright protections.\footnote{Vicky Ku, “Critique of
  the Digital Millenium Copyright Act's Exception on Encryption
  Research: Is the Exemption Too Narrow,” {\em Yale Journal of Law and
  Technology} 7 (2004): 465,
  \useURL[url12][http://heinonline.org/HOL/Page?handle=hein.journals/yjolt7&id=467&div=&collection=]\from[url12];
  Jane C. Ginsburg, {\em Legal Protection of Technological Measures
  Protecting Works of Authorship: International Obligations and the US
  Experience}, SSRN Scholarly Paper (Rochester, NY: Social Science
  Research Network, August 2005),
  \useURL[url13][http://papers.ssrn.com/abstract=785945]\from[url13];
  Aaron Perzanowski, {\em Rethinking Anticircumvention's
  Interoperability Policy}, SSRN Scholarly Paper (Rochester, NY: Social
  Science Research Network, September 2008),
  \useURL[url14][http://papers.ssrn.com/abstract=1224742]\from[url14];
  David Fry, “Circumventing Access Controls Under the Digital Millennium
  Copyright Act: Analyzing the Securom Debate,” {\em Duke Law and
  Technology Review} 2009 (2009): 1,
  \useURL[url15][http://heinonline.org/HOL/Page?handle=hein.journals/dltr2009&id=65&div=&collection=]\from[url15];
  Fred Von Lohmann, {\em Unintended Consequences: Twelve Years Under the
  DMCA} (Electronic Frontier Foundation, 2010),
  \useURL[url16][http://eric.ed.gov/?id=ED509862]\from[url16].} This
means that if an electronic book is encrypted in some way to prevent
copyright infringement, the reader may also be prevented from examining
modes of accessibility, preservation, or freedom of speech embedded into
the device.

The material affordances of text at that bottom-most, meaning-bearing
medium influence all higher-level functions of interpretation.\footnote{For
  example, see Paul Ricoeur writing on the change in media from speaking
  to writing: \quotation{The most obvious change from speaking to
  writing concerns the relation between message and its medium or
  channel. At first glance, it concerns only this relation, but upon
  closer examination, the first alteration irradiates in every
  direction, affecting in decisive manner all the factors and functions}
  Paul Ricœur, {\em Interpretation Theory: Discourse and the Surplus of
  Meaning} (Fort Worth: Texas Christian University Press, 1976), 25.}
Still, most available theories of interpretation build on properties and
assumptions attached to print media. For example, in Hans-Georg
Gadamer's seminal conception of art, the free play of the artistic mind
transforms into material structure ({\em Gebilde}) that is both
\quotation{repeatable} and \quotation{permanent.}\footnote{Hans-Georg
  Gadamer, {\em Truth and Method} (New York: Seabury Press, 1975), 110.}
Similarly, in his {\em Interpretation Theory}, Paul Ricoeur writes about
the \quotation{range of social and political changes} related to the
invention of writing. For Ricoeur, human discourse is \quotation{fixed}
and thereby \quotation{preserved from destruction} in writing.\footnote{Ricœur,
  {\em Interpretation Theory}, 26--28.} The electronic literary device
offers no such permanence. What is meant by \quotation{fixed,}
\quotation{permanent,} and \quotation{repeatable} changes with the
medium. Such properties come to us under the guise of surface
representation, which obscures the flows of code and codex. Nothing is
guaranteed in the passage of electronic text from one pair of hands into
another. Digital formatting expands its purview far beyond typographical
convention. The erasure of words, word substitution, automatic
summarization, wholesale generation of discourse by algorithmic
means---the command and control layer contains all such possibilities.
What does it mean to read and to interpret a dynamic text, which changes
depending on its context? How can literary analysis---close reading,
philology, hermeneutics---persist without the fixity of print?

Consider the commonplace task of \quotation{turning pages} in the act of
writing or reading digital texts. In cognitive linguistic terms, the
idea of paper pages should somehow extend into the domain of
manipulating digital information. In literary terms, the projection of a
page on the screen carries the tenor of paper pagination. In this way,
the turning of simulated pages implies a certain familiar arrangement of
matter. Readers know what to do with paper pages. They understand their
affordances. The metaphor encourages readers to extend their knowledge
of the physical world into the projected, virtual world. For example,
paper pages can be turned. We know they usually proceed one another,
sequentially. And we are attuned to expect the same attributes to hold
true in the vehicle---the domain receiving the tenor of the metaphor.
The action of turning virtual pages should, in theory, set off a series
of corresponding actions in the target, digital domain. In other words,
turning the page on a screen should correspond to a similar action on
the disk. But the action does not necessarily meet our expectations. The
arrangement of information stored on the disk affords different physical
actions from the arrangement of information on the page. For example, an
English-language character occupies eight bits on a disk where a print
character occupies one. The disk can tolerate millions of rewrites,
where the paper medium wears out after only a few. The paper inscription
is visible to the naked eye where the digital inscription is not.

The simulation is perhaps necessary, because the reading and writing of
digital data can involve processes far outside of everyday experience.
For example, in reading data from solid state (FLASH) memory a circuit
imparts electrical charge through quantum tunneling onto a connected
series of floating gate transistors.\footnote{Pavan et al., “Flash
  Memory Cells-an Overview”; Bez et al., “Introduction to Flash Memory.”}

\placefigure{Formal structures at the site of the inscription.
\quotation{Perspective view of a portion of a charge translating device
illustrating a preferred electrical contact arrangement. A quantum of
charge carriers, representing an information bit {[}\ldots{}{]} can be
translated along the semiconductor {[}\ldots{}{]} sweeping the minority
carriers with it. The quantum can be detected by a simple capacitive
couple, e.g., a floating gate FET.}\footnote{W. Boyle and G. Smith,
  “Information Storage Devices,” December 1974,
  \useURL[url17][http://www.google.com/patents/US3858232]\from[url17].}}{\externalfigure[images/floating-gate.png]}

Whatever the complexities of solid state storage architecture, the
difference in arrangement of information between pages and floating
gates---at the root of modern \quotation{sold state} storage---is
apparent. The structure of one has only an arbitrary connection to the
structure of the other. Consequently changes in the structure of one
domain do not necessitate changes in the structure of another: to
\quotation{erase a word} on a projected, virtual page thus may not have
the corresponding effect on the level of the storage medium. The
information may persist despite the intended erasure. As dwellers of
simulated worlds, we hope that the analogy between paper and pixel
achieves a level of verisimilitude. Turning the page or erasing a word
on the screen should do something similar on the disk. However, we also
know that not to be the case. As in Baudrillard's map, the metaphor is
broken in that it no longer reflects any terrain. The computational
metaphor simulates the familiar but absent affordances of the print
artifact. The simulation suggests a structuring of one kind, while
enacting a structure of another.

\section[tables]{Tables}

\placetable{Demonstration of simple table syntax.}
\starttable[|r|l|c|l|]
\HL
\NC Right
\NC Left
\NC Center
\NC Default
\NC\AR
\HL
\NC 12
\NC 12
\NC 12
\NC 12
\NC\AR
\NC 123
\NC 123
\NC 123
\NC 123
\NC\AR
\NC 1
\NC 1
\NC 1
\NC 1
\NC\AR
\HL
\stoptable

In these novel conditions, the task of the literary scholar must
include, among other things, a practice of microscopic reading that
corresponds to the kind of micromolecular writing suggested by
Baudrillard and Frederich Kittler.\footnote{Friedrich A. Kittler, “There
  Is No Software,” {\em CTHEORY} a032 (October 1995),
  \useURL[url18][www.ctheory.net/articles.aspx?id=74]\from[url18].} The
full extent of the simulated figure must be made available for
interpretation. What happens in the metaphorical transference between
the book as a work of art and the apparatus simulating the book?
Estrangement, the exegesis of the metaphor, reveals mechanisms of
governance shaping mental experience. It apprehends the revealed
mechanics of computational reading. Materialist poetics subsequently
allow one to consent, or, conversely, to resist elements of imposed
structure.

\reference[refs]{}%

\stopchapter
\stopcomponent