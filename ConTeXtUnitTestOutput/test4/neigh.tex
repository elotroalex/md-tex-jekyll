\setvariables[article][shortauthor={Janet Neigh}, date={03.2016}, issue={50}, DOI={}]

\setupinteraction[title={Digitizing the \quote{Sound Explosions} of Anglophone Caribbean Performance Poetry},author={Janet Neigh}, date={03.2016}, subtitle={Digitizing Sound Explosions}]
\environment env_journal


\startcomponent Explosions


\startchapter[title={Digitizing the \quote{Sound Explosions} of Anglophone Caribbean Performance Poetry}, marking={Digitizing Sound Explosions}, reference={Explosions}]


\startlines
{\bf
Janet Neigh
}
\stoplines

\startnarrower
{\it The Internet offers new performance platforms for Caribbean poets who prioritize oral expression. One group realizing this potential is the youth-led artist collective The 2 Cents Movement, based out of Trinidad and Tobago, who circulate their video poems on social networking sites. Through their DIY approach, they are building a broader youth audience for their work. More generally, Caribbean poets are often underrepresented in online poetry resources, particularly in institution-based Internet audio archives. This article uses The 2 Cents Movement as a preliminary model to understand how Internet audio archives can be redesigned to amplify the total expression of Caribbean poetry. Admittedly, The 2 Cents Movement offers a partial solution to a much bigger problem; however, this article demonstrates the value of examining poets' engagements with digital technologies to develop better archival standards and practices in sync with the politics of the work. }
\stopnarrower

\subsection[introduction]{Introduction}

If Louise Bennett, the prolific Jamaican artist, were still alive and performing today, how would the Internet fit into her creative practice? Would she post poetry recordings on Facebook? Would children play with a Miss Lou app? Would Aunty Roachy deliver wisdom via Twitter? Since she embraced emerging technologies to teach her audience about the value of the Jamaican language, social media likely would have played a prominent role in her education activism. She turned to performance early in her career to build an inclusive audience, and her media projects for the Jamaica Broadcasting Corporation--her children's television show {\em Ring Ding} (1970--82) and her radio program {\em Miss Lou's Views} (1966--82)--gave her an expanded terrain to develop her call-and-response poetics. Unfortunately, searching for Louise Bennett online today yields few actual sound recordings of her poetry. Bennett's underrepresentation online--or, more aptly, her silence, since her words can be found but rarely her voice--is characteristic of Caribbean poetry more generally. This article stems from concern about who has the power to define poetry in virtual spaces and whose voices receive airtime. As Kamau Brathwaite establishes in his groundbreaking lecture {\em History of the Voice}, Anglophone Caribbean poetry should be celebrated for its \quotation{sound explosions,} yet it has remained relatively quiet online.\footnote{Edward Kamau Brathwaite, {\em History of the Voice: The Development of Nation Language in Anglophone Caribbean Poetry} (London: New Beacon Books, 1984), 13.}

\placefigure{obligatory kitten at 300x600}{\externalfigure[http://placekitten.com/1300/1710?foo.jpg]}

Internet audio archives for poetry have flourished in recent years. Websites such as \useURL[url1][http://writing.upenn.edu/pennsound/][][PennSound]\from[url1], \useURL[url2][http://www.poetryarchive.org][][The Poetry Archive]\from[url2], \useURL[url3][http://www.poetryfoundation.org][][the Poetry Foundation]\from[url3], \useURL[url4][https://www.poets.org][][the Academy of American Poets]\from[url4], and \useURL[url5][http://www.ubuweb.com][][UbuWeb]\from[url5] have provided excellent resources for contemporary poetics. These twenty-first-century open-access collections offer both live and studio recordings, roundtable discussions of poetry, lectures, and taped interviews. While these digital platforms promise democratization, they often repeat the exclusions of print archives. Many of these collections, which now determine the content of poetry syllabi in university classrooms, reinforce the United States as the dominant center of Anglophone poetry in the Americas.\footnote{See \quotation{PennSound Transforms How Poetry Is Taught the World Over,} {\em Penn News}, 26 June 2014, \useURL[url6][http://www.upenn.edu/pennnews/news/pennsound-transforms-how-poetry-taught-world-over]\from[url6].} For example, the Poetry Foundation has 3,649 author pages, almost half of which are devoted to US poets. The site features only eleven poets from the Caribbean region, excluding many major voices from the region, including Louise Bennett. Of the eleven Caribbean poet pages, none include audio poems and only seven of them include actual poetry texts; the other four just feature short author bios. Not only are Caribbean poets underrepresented in these curated collections but their online presence is often sparse, limited to random YouTube videos.\footnote{For a discussion of race in digital canons, see Amy E. Earhart, \quotation{Can Information be Unfettered? Race and the New Digital Humanities Canon,} in Mathew K. Gold, ed., {\em Debates in the Digital Humanities} (Minneapolis: University of Minnesota Press, 2012), Open Access Edition, <\useURL[url7][http://dhdebates.gc.cuny.edu]\from[url7]>. Earhart urges that \quotation{we need to examine the canon that we, as digital humanists, are constructing, a canon that skews toward traditional texts and excludes crucial work by women, people of color, and the GLBTQ community.}}

Despite their underrepresentation in web-based audio collections, many contemporary Caribbean performance poets have embraced digital technologies to build their audiences. These new digital poetries offer a solution to what Laurence A. Breiner has characterized as the \quotation{half-life} of Caribbean performance poems.\footnote{Laurence A. Breiner, \quotation{The Half-Life of Performance Poetry,} {\em Journal of West Indian Literature} 8, no. 1 (1998): 20.} According to him, readers unable to attend a poem's live recitation are always at a disadvantage, because things like tone, gesture, and the poet's relationship with her audience also determine meaning. In terms of distribution, digitized poetry recordings, unhindered by shipping fees and trade tariffs, surpass print books in their international portability, making them more accessible to Caribbean diasporic communities.

\placefigure{2 Cents Movement Poet}{\externalfigure[poet-performing-2.jpg]}

This article examines one of the best examples of innovative digital poetry in the Caribbean: the youth-led artist collective \useURL[url8][https://twitter.com/2CentsMovement][][The 2 Cents Movement]\from[url8], based out of Trinidad and Tobago, who use social networking sites to circulate their video poems. Students at the University of the Southern Caribbean started the movement in 2010, and their activities have strengthened youth enthusiasm for spoken-word poetry. Some of their activities include island-wide school tours, open-mic nights, slam contests, television and radio broadcasts, video production of poems, and student poetry workshops. They present themselves as a movement to align their poetry with social change. Their digital recordings become modes of production, publication, distribution, preservation, and community activism. Their populist approach builds on the vernacular pedagogical dynamics of performance established by figures like Louise Bennett. By using these technologies to integrate creative and political praxes, The 2 Cents Movement realizes Brathwaite's theory of nation language and his argument that sound reproduction technologies can be used to reinvigorate Afro-centric oral traditions.\footnote{Brathwaite, {\em History of the Voice}, 13.}

Brathwaite identifies how nation language poets challenge the way that paper records were used to discredit oral ways of knowing and uphold institutional power during colonization. When a dub poet like Linton Kwesi Johnson recites definitive works such as \quotation{\useURL[url9][https://www.youtube.com/watch?v=8omA7huF6XE][][Reggae Sounds]\from[url9],} lines such as \quotation{foot-drop find drum, blood story / bass history is a moving / is a hurting black story} attack print-centric forms of knowledge.\footnote{Linton Kwesi Johnson, {\em Mi Revalueshanary Fren} (Keene, NY: Ausable Press, 2006), 15.} He builds on a history of Caribbean people using performance to circumvent the print archive to produce, record, and transmit knowledge through embodied acts.\footnote{Here I draw on Diana Taylor's analysis of what she calls the repertoire to characterize how performance acts create cultural memory in the Americas. See Diana Taylor, {\em The Archive and the Repertoire: Performing Cultural Memory in the Americas} (Durham, NC: Duke University Press, 2003).} Such knowledge depends on the interaction between the performer and her audience. The format of Bennett's {\em Ring Ding}, where she would invite children from the studio audience to join her on stage to recite poetry with her, exemplifies this well. Such communal pedagogies also extend to adult audiences. For example, Oku Onuora defines the goal of dub poetry as consciousness-raising: \quotation{It also mean to dub out the isms and schisms and to dub consciousness into the people-dem head.}\footnote{Quoted in Mervyn Morris, {\em \quote{Is English We Speaking}: and Other Essays} (Kingston: Ian Randle Publishers, 1999), 38.} Performance creates what Brathwaite characterizes as \quotation{total expression,} engendering a space for collective, politicized vernacular consciousness to take shape:

The other thing about nation language is that it is part of what may be called {\em total expression}. . . . Reading is an isolated, individualistic expression. The oral tradition on the other hand demands not only the griot but the audience to complete the community: the noise and sounds that the maker makes are responded to by the audience and are returned to him. Hence we have the creation of a continuum where meaning truly resides.\footnote{Brathwaite, {\em History of the Voice}, 18--19 (italics in original).}

Although Brathwaite seems to be arguing here that the full experience of the sonic continuum depends on live interaction, he actually emphasizes the opposite. A central point of his text, originally titled \quotation{an electronic lecture} for his 1979 Harvard presentation, harnesses the potential of sound reproduction technologies to create new pathways for total expression. For him, they intensify \quotation{the detonations within Caribbean sound-poetry {[}that{]} have imploded us into new shapes and consciousness of ourselves.}\footnote{Ibid., 49.} Brathwaite's combustion metaphors of \quotation{detonations} and \quotation{explosions} reveal his prioritizing of electronic sound. As Jonathan Sterne establishes, \quotation{All sound-reproduction technologies work through the use of transducers} that convert acoustic waves into electronic impulses.\footnote{Jonathan Sterne, {\em The Audible Past: Cultural Origins of Sound Reproduction} (Durham, NC: Duke University Press, 2003), 22.} Brathwaite suggests that these energy conversions amplify total expression rather than weaken it, which The 2 Cents Movement substantiates.

So much scholarship focuses on written documents and techniques from the past, yet many digital humanists are early adopters. They embrace the unfinished, they invite collaboration, they move us off the page, and they aim to build things for the future. Inspired by this approach, this article uses The 2 Cents Movement as a preliminary model to understand how Internet audio archives can be redesigned to amplify the total expression of Caribbean poetry. Admittedly, The 2 Cents Movement offers an incomplete solution to a much larger problem; however, I argue that we need to begin by examining poets' engagements with digital technologies to develop better archival standards and practices in sync with the politics of the work. In this collaborative spirit, I also invite readers to click on the links as they read this article, to immerse themselves in what's already online, and to think about how we can build on this virtual world of Caribbean poetics.

Part of what makes Brathwaite's {\em History of the Voice} an important catalyst for building digital archives is that he avoids the tendency in oral scholarship to fixate on the live event. For example, Hugh Hodges stresses the difficulty of studying performance poetry, lamenting that we typically examine only \quotation{the textualized trace of it.}\footnote{Hugh Hodges, \quotation{Poetry and Overturned Cars: Why Performance Poetry Can't Be Studied, (and Why We Should Study It Anyway),} in Susan Gingell and Wendy Roy, eds., {\em Listening Up, Writing Down, and Looking Beyond: Interfaces of the Oral, Written, and Visual} (Waterloo, ON: Wilfrid Laurier University Press, 2012), 98.} He considers sound and video recordings as texts, because he argues that the total expression of the live event does not get captured on them. Similarly, Peggy Phelan claims that what she calls \quotation{liveness} disappears in recordings of a performance.\footnote{As Phelan describes, \quotation{Performance's only life is in the present. Performance cannot be saved, recorded, documented, or otherwise participate in the circulation of representations of representations: once it does so, it becomes something other than performance.} Peggy Phelan, {\em Unmarked: The Politics of Performance} (New York: Routledge, 1993), 146.} As Sterne puts it, this attitude upholds \quotation{face-to-face communication and bodily presence} as \quotation{the yardsticks by which to measure all communicative activity} and \quotation{define{[}s{]} sound reproduction negatively, as negating or modifying an undamaged interpersonal or face-to-face copresence.}\footnote{Sterne, {\em The Audible Past}, 20.} Brathwaite takes a more innovative approach to sound reproduction technologies by highlighting their capacity to challenge print-centric aesthetics. The 2 Cents Movement demonstrates how digital technologies provide even more opportunities for this through their use of music videography and microblogging. However, what Brathwaite envisioned over 30 years ago through \quotation{resonating tape{[}s{]},} eight tracks, and LPs has yet to be fully realized in the broader terrain of our research methodologies and poetry resources.\footnote{Brathwaite, {\em History of the Voice}, 49.}

Critical approaches to performance poetry have been slow to develop because of the Western assumption that \quotation{the text of a poem--that is, the written document--is primary and that the recitation or performance of a poem by the poet is secondary and fundamentally inconsequential to the \quote{poem itself.}}\footnote{Charles Bernstein, ed., {\em Close Listening: Poetry and the Performed Word} (New York: Oxford University Press, 1998), 8.} While the digital should free us from our obsession with a text-based idiom in literary studies, it has yet to do so.\footnote{For a discussion about the neglect of sound analysis in digital literary studies, see Tanya Clement, David Tcheng, Loretta Auvil, Boris Capitanu, and Megan Monroe, \quotation{Sounding for Meaning: Using Theories of Knowledge Representation to Analyze Aural Patterns in Texts,} {\em Digital Humanities Quarterly} 7, no. 1 (2013), \useURL[url10][http://www.digitalhumanities.org/dhq/vol/7/1/000146/000146.html]\from[url10].} As this article demonstrates, even the designs of most online poetry audio collections still privilege the methodology of textual close reading rather than encourage us to develop new sound-based methods of analysis.\footnote{Although I do not have the space in this article to explore their work, studying the standards and practices developed by music historians for online collections such as the \useURL[url11][https://www.naxosmusiclibrary.com/home.asp?rurl=\%2Fdefault\%2Easp][][Naxos Music Library]\from[url11] would be another fruitful line of inquiry to help literary scholars move beyond text-based approaches.} In a similar vein, recent efforts to build Caribbean digital archives have focused on converting print documents and have not engaged as much with sound media.\footnote{While there are Caribbean sound archives, such as \useURL[url12][http://radiohaitilives.com][][RadioHaiti]\from[url12], in Caribbean literary studies print archiving is much more prevalent. For example, major collections, such as the \useURL[url13][http://www.dloc.com][][Digital Library of the Caribbean]\from[url13], while excellent, primarily hold literary materials converted from print documents.} Resources and funding certainly drive this, since print materials are easier and cheaper to digitize. Yet given the oral dimensions of Caribbean cultures, we must address how to represent and archive performance to overcome the structural biases of print archives.

\subsection[creating-digital-griots-the-2-cents-movement]{Creating Digital Griots: The 2 Cents Movement}

\placefigure{Audience Interaction}{\externalfigure[audience-interaction-2.jpg]}

Scrolling through \useURL[url14][https://www.facebook.com/The2CentsMovement/][][The 2 Cents Movement's Facebook page]\from[url14] reveals a vibrant online poetry community. As of December 2015, they have more than 13,000 followers, and their page includes announcements about upcoming performances, photographs of performances, political news updates, links to live-stream their events, and video poems. Its campus origins explain why Facebook, a social networking site designed for college students, has played a central role in the development of the movement. The founder, Jean Claude Cournand, an undergraduate at the time, sought to build a stronger intellectual youth culture--to invite youth to put in their two cents by expressing their views on current issues (ranging from marijuana legislation to homophobia) through spoken-word poetry. In an interview, he explained that they chose \quotation{to go into the digital habitat rather than trying to reach people through events alone,} in order to attract a youth audience.\footnote{Quoted in Bobie-lee Dixon, \quotation{Not afraid to put in their \quote{2 cents,}} {\em Trinidad and Tobago Guardian}, 16 July 2013,

  \useURL[url15][http://www.guardian.co.tt/entertainment/2013-07-15/not-afraid-put-their-‘2-cents’][][http://www.guardian.co.tt/entertainment/2013-07-15/not-afraid-put-their-\quote{2-cents}]\from[url15].} Their best-known projects are the Intercol and Verses national slam competitions, held annually as part of the Bocas Lit Festival (Verses was recently renamed the First Citizens National Poetry Slam). The two different slams--Intercol features high school teams, and Verses features prominent poets who compete individually--reflect their mentorship structure. The 2015 winner of the Intercol slam, \useURL[url16][http://www.bocaslitfest.com/2015/the-spoken-word-intercol-champion-is/][][Michael Logie]\from[url16], got to pick a 2 Cents Movement poet as a mentor to work with him for a year to help him develop his poetry.\footnote{\quotation{Logie Tops Schools Spoken Word \quote{Intercol,}} {\em Trinidad and Tobago Guardian}, 2 April 2015,

  \useURL[url17][http://www.guardian.co.tt/lifestyle/2015-04-01/logie-tops-schools-spoken-word-‘intercol’][][http://www.guardian.co.tt/lifestyle/2015-04-01/logie-tops-schools-spoken-word-\quote{intercol}]\from[url17].} The 2 Cents Movement has also collaborated with the Trinidad and Tobago Radio Network on the \useURL[url18][https://www.youtube.com/playlist?list=PLsZJUoc_yr1ufNZGIsFg74Fl3yuZTNU9k][][Free Speech Project]\from[url18], where young artists recite their poems weekly on the radio. These poems are also produced as videos and archived on the network's YouTube channel. The majority of the movement's activities are coordinated and advertised online, and major events like the slam contests can be live-streamed. They exist solely through freely accessible sites and do not have their own server or domain name, which suggests that a lack of resources does not need to be an obstacle to building an online presence for Caribbean poetry. In addition to reaching youth in Trinidad and Tobago, they have gained an international following by circulating their poetry on social media.

Their Twitter and Facebook updates for their \useURL[url19][http://www.bocaslitfest.com/courts-bocas-speak-out/][][2014--2015 Courts Bocas Speak Out Tour]\from[url19] of more than fifty secondary schools demonstrates their use of these platforms to enhance audience interaction. During this tour, some of the nation's best performance poets mentored high school students as they wrote and performed their own works through performances and workshops. These school visits provided a way for The 2 Cents Movement to promote the Intercol slam competition. Photo live-tweets of their tour performances often focused on the students actively participating in the audience rather than on the poet performing. In these shots, the camera points at the audience, often catching only the performer's back or the corner of her shoulder.\footnote{See their Twitter page: \useURL[url20][https://twitter.com/2CentsMovement]\from[url20]. See also their individual albums on their Facebook page, where some of their school tour photos are collected. For example, see the album \quotation{Five Rivers Sec -- Speak Out Tour 2014 -- Day 27}: \useURL[url21][https://www.facebook.com/media/set/?set=a.610565055739844.1073741948.125245237605164&type=3]\from[url21].}

\placefigure{Audience interaction}{\externalfigure[audience-interaction-1.jpg]}

\placefigure{Poet performer}{\externalfigure[poet-performing-1.jpg]}

The online viewer gets to stare out at the auditorium filled with students, which allows her to occupy the performer's gaze but also view a student's face as a mirror, inviting her to identify with both the performer and the audience. This elaborates on the pedagogy offered during the tour, where the poets' performances provide students with a model for their own poetry aspirations. Visually, Twitter and Facebook followers become situated in the continuum of total expression--in the reciprocal exchange between performer and audience.

\placefigure{Audience interaction}{\externalfigure[audience-interaction-3.jpg]}

High school students who attended a 2 Cents Movement performance at their own school could then follow them on Twitter and Facebook and connect these images to their own participation in the live event. The photos (and the movement as a whole) downplay individual author celebrity in favor of building a student-focused collectivity. The 2 Cents Movement uses Twitter and Facebook to create \quotation{a constancy of presence} to link together different high school communities in Trinidad and Tobago.\footnote{Dhiraj Murthy, {\em Twitter: Social Communication in the Twitter Age} (Cambridge, UK: Polity Press, 2013), 39.}

Although these social media platforms governed by the commercial needs of US-based multinationals certainly hold limitations, rhetoric scholar Kevin Adonis Browne stresses the possibility for subversive adaptation. Examining Caribbean Internet users' chatting, blogging, and video-sharing practices, he illustrates how users adapt everyday vernacular practices to assert their presence in digital public spaces that often render them invisible. While his argument relates to individuals' everyday encounters online rather than large-scale digital projects, the parallels he establishes between \quotation{a carnivalesque imperative} and the Internet lend insight into The 2 Cents Movement's adaptation of social media to expand on the power of total expression in online environments.\footnote{Kevin Adonis Browne, {\em Tropic Tendencies: Rhetoric, Popular Culture, and the Anglophone Caribbean} (Pittsburgh: University of Pittsburgh Press, 2013), 129.} Rather than view Caribbean folk cultures in opposition to digital forms, Browne views the inventiveness of both as part of what makes them compatible. He demonstrates how communal vernacular practices easily translate to an online environment because they emerge out of \quotation{a history of adaptability} necessary for \quotation{those who have had to find a way or make one when none seemed available.}\footnote{Ibid., 134.} For example, in the colonial era, musicians' invented steel pan drumming to get around the ban on using actual drums and managed to make noise in a colonial environment that aimed to silence them. Corporate social media platforms also curtail creative expression through their circumscribed structures. Such restrictions often reinforce the majority voice, yet Facebook and Twitter have also proven themselves useful for the growth of subcultures and movements.

Media scholars have identified how Facebook's interface encourages users to adopt neoliberal values. The anthropologist Alex Fattal draws a connection between a user's desire to gain as many likes and friends as possible and Facebook's corporate desire to become the most dominant social network in the world, characterizing this as \quotation{the unspoken logic of accumulation and curiosity that undergirds} the platform.\footnote{Alex Fattal, \quotation{Facebook: Corporate Hackers, a Billion Users, and the Geo-Politics of the \quote{Social Graph,}} {\em Anthropological Quarterly} 85, no. 3 (2012): 931. See also Ilana Gershon, \quotation{Un-Friend My Heart: Facebook, Promiscuity, and Heartbreak in a Neoliberal Age,} {\em Anthropological Quarterly} 84, no. 4 (2011): 865-94.} To what extent can this \quotation{logic of accumulation and curiosity,} which sounds eerily similar to older imperial impulses, be disrupted? By using social media to augment their performance events, The 2 Cents Movement intercepts this logic of accumulation by bringing users into the dynamic of total expression. While scholars often critique Facebook for weakening face-to-face sociality, The 2 Cents Movement reveals how it can be used to deepen social connections.\footnote{For a critique of Facebook as a community building form, see José Marichal, {\em Facebook Democracy: The Architecture of Disclosure and the Threat to Public Life} (Abingdon, Oxon, UK: Ashgate, 2012).}

Through social media, they also make their performances part of everyday life. People browse Facebook as they ride the bus to work, or as they wait in line at the grocery store. The 2 Cents Movement often depicts the performance dynamics of such quotidian spaces in their studio-produced video poems available on YouTube. For example, their most popular video poem \useURL[url22][https://www.youtube.com/watch?v=9OA8eLoPylU][][\quotation{Maxi Man Tracking School Gyal}]\from[url22] (originally titled \quotation{Yankee's Gone}), by Crystal Skeete, takes place on a Curepe street and on a bus. Skeete addresses her concern about adult males sexually preying on underage schoolgirls by writing a poem in the voice of a teenage girl who refuses their advances. Similar to earlier women poets like Miss Lou and Jean \quotation{Binta} Breeze, she uses the dramatic monologue to give voice to women who are denied respect in the public sphere. Through social media her video becomes even more impactful because a viewer might be in that particular social space as she watches the poem and even be witnessing what Skeete describes. This restores an important aspect of Caribbean oral traditions, where yard performances blend into everyday practices rather than become isolated special events. Miss Lou had a similar aim in her market women poems, yet her ability to connect her audience to her settings was often restrained by the artifice of the theater stage. By creating a tangible bond between their real-life users and the mediated voices in their poems, The 2 Cents Movement invites users to do more than merely like their status updates and continue scrolling.

In addition to microblogging their performances, The 2 Cents Movement's video poems, like Skeete's \quotation{Maxi Man,} demonstrate how audiovisual recording technologies can be used to enhance the communal dynamic of total expression. Their videos challenge the idea that poetry depends on print publication, since many of their well-known poems are available only in this format. By getting us off the page, their videography expands the possibilities for how we might edit and archive performance poetry. Released in 2013, Skeete's video went viral on social media and received more than 88,000 views on YouTube.\footnote{Crystal Skeete, \quotation{Maxi Man Tracking School Gyal,} YouTube video, 5.27, posted by The 2 Cents Movement, 25 June 2013, \useURL[url23][https://www.youtube.com/watch?v=9OA8eLoPylU]\from[url23].} Her poem exemplifies the 2 Cents Movement's approach, in that she initially attended one of their workshops as a medical student and she is now a leading poet in the movement. She won the 2013 Verses poetry slam with her performance of this poem; however, the popular version on The 2 Cents Movement's YouTube channel is a studio recording made by BeatOven Productions with videography by Denithy. Rather than aim to recreate the experience of the live performance event, the video invites a viewer to feel included in the street scene being depicted. Through their synthesis of the dramatic monologue form and the music video genre, Skeete and the 2 Cents Movement team cultivate a sense of audience connection and encourage dialogue beyond the space of the poem.

\placefigure{Maxi man cast}{\externalfigure[maxi-man-cast.jpg]}

Skeete dresses in a government school uniform to play her character in the video. Her embodiment of the character creates a dialogic relationship between her persona and her words because her recitation critiques the behavior that she acts out. The dramatic monologue's implied addressee is another high school girl who is swayed by a maxi man's advances. The speaker warns her that \quotation{these free rides are going to end in horror / trust me / with that nine-month sentence things does go sour.}\footnote{Skeete, \quotation{Maxi Man} (hereafter cited in text). Since I transcribed the quotations from a recording, any inaccuracies are mine.} Skeete's conversational style, enhanced by the video editing and camera angles, encourage the viewer to feel as though Skeete is speaking directly to her. Standing in the middle of the street, Skeete begins with the lines, \quotation{Posing on every street corner / this is the resurrection of \quote{Jean and Dinah}\ldots{} No Yankees here / only maxi man conductor.} Referencing Mighty Sparrow's calypso hit \quotation{Jean and Dinah,} about how the presence of a World War II US army base led to prostitution in Trinidad, she draws attention to how girls getting free bus rides from maxi men for sexual favors recreates a similar dynamic. Through an intergenerational call and response, she situates her poem in the Trinidadian oral tradition. Skeete's poem exemplifies how The 2 Cents Movement aims to cultivate a socially conscious form of learning that values vernacular language and knowledge.

Similar to music videos, Skeete's recitation alternates between a voiceover and her character speaking onscreen. By playing with the boundaries between diegetic and non-diegetic sound, the editing capitalizes on sound's permeability, encouraging her virtual audience to feel a closer connection to her words. Skeete's voice extends beyond the boundaries of the video and infiltrates the listening space, collapsing the distance between the onscreen performance and the audience's experience of it, especially if they happen to be in a similar setting.

In the narrative that unfolds, the camera angles accentuate Skeete's embodied performance. While conventional music videos often sexually objectify women's bodies, the jump cuts between medium shots, medium long shots, and close-ups (often from a slight low angle) emphasize Skeete's power as she dramatizes her character's self-actualization and encourages other girls to do the same. Skeete's words combined with the camera work challenge the male gaze to create a virtual space of female community. This reaches its peak when the speaker sits in the back of the van reciting her critique and a reverse shot depicts the maxi man gazing at her in the rearview mirror. His gaze is literally reflected back at him, while the speaker verbally refuses his objectification. In the middle of the van, a girl flirts with another maxi man. Throughout the poem, the speaker seems to completely condemn this behavior; however, the poem ends with a surprise twist. As she gets out of the maxi man's van, she tells the audience, \quotation{This brown-skinned gyal / going home and mind she child,} indicating that her advice is based on her own mistakes with a maxi man (Skeete). This surprise ending builds the speaker's authority (not by separating her from her community but by underscoring her participation in it) and encourages schoolgirls watching the poem to trust her advice.

In the live performance of \quotation{Maxi Man Tracking School Gyal,} Skeete's critique of sexual harassment is likely powerful; however, the video format allows a viewer to experience with greater clarity the embodied experience of resisting sexual harassment on the street. For female viewers who went through something similar, the recreation of the scene and its provocative ending deepens their identification with the speaker. Like a griot, Skeete speaks for a community of young girls, or rather speaks with them, as her performance encourages them to bring their experiences into their interpretation of the poem.

Unlike many poetry recordings available online, which are taped at live events, The 2 Cents Movement engages technologies as part of the artistic composition of the poem through their studio-produced poems. By borrowing from music videography, they invite us to see performance archives as something more expansive than as a place to simply store copies of \quotation{original} live events. This moves us move beyond what Sterne characterizes as the preoccupation with fidelity, which advertisers established early on in sound reproduction history. Sterne reveals how \quotation{sound fidelity is a story that we tell ourselves to staple separate sonic realities together.} It maintains the illusion of \quotation{reproduced sound as a mediation of \quote{live} sounds;} however, \quotation{the \quote{medium} does not necessarily mediate, authenticate, dilute, or extend a preexisting social relation.}\footnote{Sterne, {\em The Audible Past}, 219, 218, 226.} Rather than only view an archive as a space to preserve former instances of total expression, we should also view them as an opportunity to engender new forms of it. Rather than use recording technologies to recover or preserve a lost history of the voice, they can be used to actively construct a history of the voice--to create a sonic reality where new social relations can take shape.

\subsection[redesigning-internet-audio-archives-for-caribbean-poetry]{Redesigning Internet Audio Archives for Caribbean Poetry}

The 2 Cents Movement's DIY approach to digitizing poetry illustrates that a lot can be accomplished by adapting existing platforms and with limited resources; however, the underrepresentation of Caribbean poetry online will not be solved solely through social media and the \quotation{free culture} of the Internet. Merely uploading more poems to YouTube will not be enough to overcome the structural biases that silence (and drown out) Caribbean performance poetry online. As Rick Prelinger establishes, YouTube has become \quotation{in the eyes of the public, the default online moving-image archive.}\footnote{Rick Prelinger, \quotation{The Appearance of Archives,} in Pelle Snickars and Patrick Vonderau, eds., {\em The YouTube Reader} (Stockholm: National Library of Sweden, 2009), 269.} Yet it lacks many of the defining characteristics of archives that make them valuable, such as a focus on long-term preservation and \quotation{strictly codified lines of conduct} that have been carefully developed through academic practice and intellectual debate.\footnote{Frank Kessler and Mirko Tobias Schäfer, \quotation{Navigating YouTube: Constituting a Hybrid Information Management System,} in Snickars and Vonderau, {\em The YouTube Reader}, 277.} The underrepresentation of Caribbean poetry in Internet audio archives is also symptomatic of the global digital divide, which is especially acute for Africa-descended populations.\footnote{Mary F. E. Ebeling, \quotation{The New Dawn: Black Agency in Cyberspace,} {\em Radical History Review} 87 (Fall 2003): 98. For a discussion of the global digital divide, see Pippa Norris, {\em Digital Divide: Civic Engagement, Information Poverty, and the Internet Worldwide} (New York: Cambridge University Press, 2001).} However, Curwen Best points out that \quotation{the flip side to the notion of the digital divide is therefore the formation of strategic space. It is this strategic space that opens up an area and arena of knowledge about evolving technologies.}\footnote{Curwen Best, {\em The Politics of Caribbean Cyberculture} (New York: Palgrave Macmillan, 2008), 4.} To take advantage of \quotation{this strategic space,} this section of the article evaluates the standards and practices that have been established by existing Internet audio archives and considers how a Caribbean poetry collection could be organized to realize total expression.

A discussion about archiving poetry recordings would be incomplete without a consideration of the achievements of \useURL[url24][http://writing.upenn.edu/pennsound/][][PennSound]\from[url24]. As the leading US collection, it has quickly become the standard for how to create a web-based poetry audio archive. University of Pennsylvania professors Charles Bernstein and Al Filreis started the collection in 2003 primarily for classroom use. It holds over 45,000 poetry recordings available for free streaming and download. Their files are downloaded roughly four million times a month.\footnote{Tanya Barrientos, \quotation{Penn's Rich Poetry Legacy,} {\em Penn Current}, 20 May 2010, \useURL[url25][http://www.upenn.edu/pennnews/current/2010-05-20/features/penn’s-rich-poetry-legacy]\from[url25].} For poetry, this number is astounding when one considers the limited print runs and book sales for contemporary poetry. Although US poets and avant-garde aesthetics tend to predominate, the site includes author pages for some prominent Caribbean poets, including \useURL[url26][http://writing.upenn.edu/pennsound/x/Brathwaite.php][][Brathwaite]\from[url26], and \useURL[url27][http://writing.upenn.edu/pennsound/x/Philip.php][][M. NourbeSe Philip]\from[url27]. They also hold some of the few \useURL[url28][http://writing.upenn.edu/pennsound/x/Bennett.php][][Bennett]\from[url28] recordings available online. The site uses the following archival taxonomies: authors, series, anthologies, collections, groups, and classics, which provide the user with a variety of ways to study a poet's work. The \useURL[url29][http://writing.upenn.edu/pennsound/manifesto.php][][PennSound Manifesto]\from[url29] is worth examining because it has set the precedent for subsequent Internet poetry collections. They provide the following criteria for poetry audio: \quotation{It must be free and downloadable;} \quotation{It must be MP3 or better;} \quotation{It must be singles;} \quotation{It must be named;} \quotation{It must embed bibliographic information in the file;} and \quotation{It must be indexed.}\footnote{Charles Bernstein, \quotation{PennSound Manifesto,} PennSound, 2003, \useURL[url30][http://writing.upenn.edu/pennsound/manifesto.php]\from[url30].} These criteria have proved successful in terms of encouraging widespread use, particularly on college campuses. They archive each poem individually (rather than entire events), because this makes them easier to find, access, and download, and it allows listeners to create their own playlists. In their own words, they adapt \quotation{a consumer-oriented MP3 file exchange approach} for a non-profit library.\footnote{Ibid.} By embedding bibliographic information in the file and naming each one, they uphold archival standards and ensure that future researchers will have access to information about the context for the recording, something a social media platform like YouTube does not always provide. Their straightforward interface, which lets users select from an alphabetized list of authors, also erases implicit hierarchies that come from the emphasis on accumulation in social media sites where voices are forced to compete with each other for likes and views.

While Penn Sound's open access model holds a lot of potential for Caribbean performance poetry, their prioritizing of free resources may not be as easy to achieve for Caribbean poetry. An extreme example of this free culture ethos is represented by Kenneth Goldsmith, the founder of \useURL[url31][http://www.ubuweb.com][][UBUweb]\from[url31], who for awhile maintained a page on the site called the \quotation{Wall of Shame,} where he would publically condemn any artist who asked for her work to be removed from the site by writing her name on the wall.\footnote{Astra Taylor, {\em The People's Platform: Taking Back Power and Culture in the Digital Age} (New York: Picador, 2014), 153.} Unlike Goldsmith, PennSound strives to ensure that all of their recordings are \quotation{cleared for copyright to be distributed free for noncommercial and educational purposes.}\footnote{Bernstein, \quotation{PennSound Manifesto.}} While this democratic and anti-capitalistic approach may seem appealing for Caribbean poetry, in practice it may be more difficult to implement, particularly for poets who view performance as a central part of their artistic production. In the manifesto, Bernstein states that the debates about file sharing in the music industry do not apply to poetry: \quotation{One of the advantages of working with poetry sound files is that we don't anticipate a problem with rights. At present and in the conceivable future, there is no profit to be gained by the sale of recorded poetry.}\footnote{Ibid.} Yet Caribbean poets such as Mutabaruka and Linton Kwesi Johnson built their careers through their affiliation with the reggae recording industry and make money off of their albums. The 2 Cents Movement offers their recordings for free; however, as a group of younger poets, they may be more willing to share their work online to build their reputation.

The subject of copyright and fair use for Caribbean performance poetry is a complex issue, which deserves its own treatment in a separate article. In terms of the current discussion, it offers another example of the persistence of print-centrism in poetry scholarship. The emphasis on free recordings implicitly upholds the idea that the artistic labor of poetry is more worthy of payment when it is print-based. Moreover, it demonstrates how we define poems as texts rather than as performances. The Poetry Foundation's \useURL[url32][http://www.poetryfoundation.org/downloads/FairUsePoetryBooklet_singlepg_2.pdf][][{\em Code for Best Practices in Fair Use For Poetry}]\from[url32] underscores this point. A group of poets, editors, and publishers met at the Poetry Foundation's Harriet Monroe Poetry Institute and collaborated with American University's Center for Social Media and its Washington College of Law in 2011 to create the guide. Although they outline their purpose as \quotation{identifying obstacles preventing poetry from coming fully into new media and, where possible, imagining how to remove or mitigate these obstacles,} almost none of their examples involve poetry audio.\footnote{{\em Code for Best Practices in Fair Use For Poetry} (Center for Social Media and the Poetry Foundation, 2011), <\useURL[url33][http://www.poetryfoundation.org/foundation/bestpractices]\from[url33]>.} Yet for Caribbean poets a major stumbling block to \quotation{coming fully into new media} is a clear set of fair use practices for performance-based work. In the \quotation{Poetry Online} section, their examples pertain to the appearance of text on websites. The document concludes with a section on \quotation{Literary Performance;} however, it only addresses poets who incorporate poems by other artists into their readings. This document becomes completely irrelevant for groups like The 2 Cents Movement who rarely produce text versions of their poems.

This print-centrism also extends to the design of audio collections. I agree with Kate Eichhorn, who argues that most poetry sound archives (both digital and analog) have yet to realize the potential of archiving sound due to \quotation{the widely held assumption {[}in literary studies{]} that the archive is necessarily a space of writing and, hence, opposed to speech and other performative acts.}\footnote{Kate Eichhorn, \quotation{Past Performance, Present Dilemma: A Poetics of Archiving Sound,} {\em Mosaic} 42 (2009): 187.} Eichhorn proposes that we need \quotation{to create a sound archive designed to recover and preserve the embodied, interactive, and present nature of the performed word.}\footnote{Ibid., 190.} In her assessment, archived poetry recordings too often become \quotation{flat and lifeless artifacts.}\footnote{Ibid., 184.} Although Eichhorn focuses on avant-garde poets, her arguments apply to Caribbean poets who foreground sound as an integral part of the poem's meaning. Such work accessible in an archive designed to disrupt our text-based conceptions of archival knowledge would undoubtedly lead us more assertively into other forms of knowing rather than allow the digital to reconsolidate print paradigms.

One notable exception to the underrepresentation of Anglophone Caribbean poets in Internet audio collections is \useURL[url34][http://www.poetryarchive.org/][][The Poetry Archive]\from[url34], supported by the Arts Council of England. This site features more Caribbean poets than most, because \useURL[url35][http://caribbeanpoetry.educ.cam.ac.uk/][][the Caribbean Poetry Project]\from[url35]--a group of scholars and poets from the Cambridge University Faculty of Education, the Centre for Commonwealth Education, and the University of the West Indies--collaborated with The Poetry Archive to improve access. Out of 476 poets, twenty-three are from the Caribbean region. While this is not a huge percentage, it is considerably more than most sites, and every Caribbean poet page has sound recordings that can be streamed for free or downloaded for a fee of £0.89. The Poetry Archive, like many Internet audio archives, specializes in classroom resources. They have a special page for \useURL[url36][http://www.poetryarchive.org/collection/caribbean-poetry-0][][teaching Caribbean poetry]\from[url36] with a selection of sound recordings and \useURL[url37][http://www.poetryarchive.org/articles/guide-language-caribbean-poetry][][\quotation{A Guide to the Language of Caribbean Poetry.}]\from[url37] The Caribbean Poetry Project's work with The Poetry Archive demonstrates how cross-disciplinary and cross-institutional collaborations help to break down Anglo-centrism.

Yet analyzing the design of the Poetry Archive author pages reveals that we need to do more than add voices to existing archives. As Amy Earhart cautions, for digital recovery projects for writers of color to be successful, we also must \quotation{theorize the technological with the same rigor as we theorize the content.}\footnote{Earhart, \quotation{Can Information be Unfettered?}} If one compares the recording of Linton Kwesi Johnson reciting \useURL[url38][http://www.poetryarchive.org/poem/di-great-insohreckshan][][\quotation{Di Great Insohreckshan}]\from[url38] on The Poetry Archive with a \useURL[url39][https://www.youtube.com/watch?v=IUNQS7Pwwu4][][YouTube recording]\from[url39] of him performing it at an outdoor festival in Venezuela in 2008, the total expression feels much more resonant in the YouTube version than on the Poetry Archive site.\footnote{Linton Kwesi Johnson, \quotation{Di Great Insohreckshan,} YouTube video, 2.00, posted by Oscar David De Barros, 25 January 2013, \useURL[url40][https://www.youtube.com/watch?v=IUNQS7Pwwu4]\from[url40].} I make this comparison to illustrate how the design of the Poetry Archive page mutes the insurrectionary tone of Johnson's poem about the Brixton Riots, not to demonstrate that the YouTube platform is inherently better at representing total expression. Obviously, the YouTube video allows one to see Johnson's body and facial expressions, which one cannot get from the Poetry Archive audio recording, but it is not as simple as video versus audio. Although the sound quality lacks the clarity of the Poetry Archive recording, the uneven audio levels provide a more authentic experience of what hearing this poem in a large crowd at an outdoor festival would have felt like. The spontaneity of live performances, including sound glitches and background noises, becomes part of what Brathwaite characterizes as the \quotation{sonority contrasts} of total expression.\footnote{Brathwaite, {\em History of the Voice}, 46.} As Eichhorn points out, \quotation{these intruders are precisely what sound technicians often seek to filter out as they prepare recordings for the archive.}\footnote{Eichhorn, \quotation{Past Performance, Present Dilemma,} 190.} The Poetry Archive version of \quotation{Di Great Insohreckshan} follows the predominant sound-editing style for Internet audio archives. Martin Spinelli describes this style as \quotation{the seamless edit,} designed to highlight the poet's voice and minimize all other distractions, including the recording scene (whether live or in-studio) and the \quotation{material elements of their production.}\footnote{Martin Spinelli, \quotation{Analog Echoes: A Poetics of Digital Audio Editing,} {\em Object 10: Cyberpoetics} (2002): 36, {\em UbuWeb Papers}, 16 May 2007, \useURL[url41][http://www.ubu.com/papers/object/06_spinelli.pdf]\from[url41]. Spinelli criticizes poetry sound editing for being too influenced by traditional radio theory, questioning why most archived poetry recordings present sound as though it were a linear medium by using \quotation{the seamless, invisible, inaudible edit which dislodges nothing, which interrupts nothing, which is in fact deployed to remove interruption, to remove digression and to clarify} (36).}

In addition to using conventional sound-editing, \useURL[url42][http://www.poetryarchive.org/poet/linton-kwesi-johnson][][Johnson's author page]\from[url42] follows the standard visual design that can be found on The Poetry Archive as well as other online poetry collections such as PennSound. His page features an author portrait (rather than a performance action shot), a biography, and links to individual poem tracks. Beneath the poem title and author name, one can click the play button and listen to the recording. The Poetry Archive uses a simple audio player that allows a listener to start, pause, and stop the track. Johnson's author biography explains that his poems on the site were recorded from live performances and come from his CD {\em LKJ A Cappella Live}. Despite the claim that \quotation{the energy of his live recitals gives the recordings a unique electricity, interspersed with the laughter and applause of audiences around Europe,} the seamless editing of \quotation{Di Great Insohreckshan} allows one to listen to Johnson's entire recitation before realizing that it is a live performance.\footnote{Linton Kwesi Johnson Author Page, The Poetry Archive, accessed 23 July 2015, \useURL[url43][http://www.poetryarchive.org/poet/linton-kwesi-johnson]\from[url43].} Once the poem ends, the only audible audience noises are a cough and polite applause that slowly fades out. Rather than convey \quotation{a unique electricity,} the cough suggests a subdued audience who tried to remain silent until the end of the performance. The page provides no information about the specifics of the event or who the European audience was, although it sounds like it might be a poetry reading in an academic setting. The track certainly holds no trace of a communal \quotation{dub consciousness.}\footnote{Onuora quoted in Morris, {\em \quote{Is English We Speaking,}} 38.} By not clearly identifying the track as a single performance, the archive presents it as the authoritative audio version. In the left-hand margin, under the heading \quotation{About the Poem,} it lists the themes as \quotation{social, unrest, race, and Caribbean} but provides no information about the Brixton Riots.

Derek Furr notes that while online poetry collections devote attention to different aesthetics, what unites them is the idea that hearing a poet voice her own work is crucial to understanding the poem. Yet despite the interest in vocalization, collections tend to downplay the setting of the poet's performance by providing little information about when and where the reading occurred and why this particular event was recorded.\footnote{Derek Furr, {\em Recorded Poetry and Poetic Reception from Edna Millay to the Circle of Robert Lowell} (New York: Palgrave Macmillan, 2010), 5.} While certain sites, such as PennSound, make sure that all of the bibliographic info is embedded in the single file with accurate metadata, often all that this provides is the date and physical location of the reading. By divorcing the poem from its contextual setting, this approach erases the reciprocal relationship between the poet and her audience. Furr proposes that poetry recordings (both live and in-studio) become valuable because \quotation{when we close listen, we hear not only the sounds of the poem and the poet's voicing of them, but also the echoes of previous scenes of reading and listening.}\footnote{Ibid., 149.} Through webpage design and sound editing, The Poetry Archive downplays the previous audiences of \quotation{Di Great Insohreckshan} so that the online user feels as though she is the only person in the audience. While The 2 Cents Movement illustrates how cinematic approaches can be used to deepen total expression, the Johnson example illustrates the value of preserving the ephemeral qualities of a live performance in a recording to enrich the listening experience. Furr borrows Charles Bernstein's term {\em close listening} to characterize how a user should engage with audio recordings. Based on his experience with PennSound Bernstein proposes close listening as an alternative to close reading, where one prioritizes the materiality of sound and the aural experience.\footnote{See Bernstein, {\em Close Listening}, 3--26.} Yet, in practice, how much does close listening actually deviate from close reading on Internet audio archives?

Annie Murray and Jared Wiercinski, curators of \useURL[url44][http://spokenweb.ca/][][SpokenWeb]\from[url44], a collection of recordings of a Montreal poetry reading series from 1966 to 1977, point out that while most Internet audio archives make listening the focus, their structure makes them multimodal. Accordingly, they stress that the visual elements of online poetry archives need careful consideration. They explore \quotation{what kinds of site navigation, audio visualization, design elements and functionalities could be offered by a Web-based spoken word interface, and how these might enhance the listening process and, ultimately, the scholarly endeavor.}\footnote{Annie Murray and Jared Wiercinski, \quotation{Looking at Archival Sound: Enhancing the Listening Experience in a Spoken Word Archive,} {\em First Monday} 17, no. 4 (2012), \useURL[url45][http://firstmonday.org/ojs/index.php/fm/article/view/3808/3197]\from[url45].} Because very little scholarship exists on how people engage with sounded poems, they acknowledge that their suggestions for \quotation{a sound archive \quote{recipe} that other cultural heritage institutions can follow} are based on established reading practices.\footnote{Annie Murray and Jared Wiercinski, \quotation{A Design Methodology for Web-based Sound Archives,} {\em DHQ: Digital Humanities Quarterly} 8, no. 2 (2014), \useURL[url46][http://digitalhumanities.org:8081/dhq/vol/8/2/000173/000173.html\#p6]\from[url46].} They make some helpful suggestions, including using a waveform display for sound visualization, and providing listeners with a media player that allows them more control over the playback. They also suggest incorporating any available images and videos of the performance event, much like The 2 Cents Movement's photo documentation on Twitter and Facebook. However, other suggestions such as tethering audio playback with a written transcript keep the primary focus on reading rather than listening.\footnote{For a summary of their specific suggestions for visual design, see Murray and Wiercinski, \quotation{Looking at Archival Sound.}} For example, on the Poetry Archive site, beneath the link to play \quotation{Di Great Insohreckshan,} one can also click on a \quotation{Read this Poem} link. This encourages scholars to engage in what is, for many, the more familiar interpretive practice rather than to develop new modes of analysis based on sound. In digital archives, Bernstein suggests that \quotation{poems, set adrift from their visual grounding in alphabetic texts, might begin to resemble the songs from which, for so long, they have been divided.}\footnote{Charles Bernstein, \quotation{Making Audio Visible: The Lessons of Visual Language for the Textualization of Sound,} {\em Textual Practice} 23, no. 6 (2009): 966, doi:10.1080/09502360903361550.} However, to achieve this we need to 1) be wary of a one-size-fits-all approach, and 2) carefully design visual elements to encourage sonic engagement rather than to reify a visual economy of text. Such a divide between song and poem has never existed for Anglophone Caribbean performance poets, yet colonial legacies continue to dictate that we read rather than listen to this work.

Even when one is only listening, Internet audio archives often recreate the feeling of reading alone. Clicking on the link and listening to the poem while one looks at the author portrait (similar to the style found on book jackets) mimics the experience of solitary reading, where one feels in private conversation with the author. Simple audio players that have only a linear time-lapse bar (rather than a more complex waveform display) encourage the eyes to move from right to left, as they do in the act of reading. When a poem is presented as a decontextualized single track, a listener is encouraged to adopt a new critical approach and treat the poem as a self-contained object. Listening, like reading, becomes \quotation{an isolated, individualistic expression} rather than a communal endeavor.\footnote{Brathwaite, {\em History of the Voice}, 18.} Sound reproduction technologies have also encouraged our hearing to become more individualistic. Using headphones as his example, Sterne argues that such devices encourage listening to be \quotation{more orientated toward constructs of private space and private property,} which encourages \quotation{sound to become a commodity.}\footnote{Sterne, {\em The Audible Past}, 24.}

In contrast, The 2 Cents Movement recreates the communal experience of listening to a poem, which is a key part of total expression. Their photo tweets and poetry videos make online members a part of their participatory audience. We have much work to do to figure out how to build Internet audio archives that recreate this communal experience of listening. Everything--including sound editing, interface, database design, navigation system and copyright issues--needs careful consideration to realize the digital potential of total expression. To embark on this work, it is crucial that we turn to the poets themselves and build on their education legacies.

So much of the history of the voice has already been lost, or muted, by the ongoing colonial bias toward nation language, as well as limited resources and funding. Caribbean poetry recordings that survive buried in dusty boxes in attics and library storage rooms sit silently waiting for new audiences to return to them \quotation{the noise and sounds that the maker makes.}\footnote{Brathwaite, {\em History of the Voice}, 18--19.} Since these reel-to-reel tapes, cassettes, eight-tracks, CDs, and LPs are fragile media, threatened by technological obsolescence and (sometimes) the tropical Caribbean climate, we must make digitizing this vital record of Caribbean poetry history a priority.

\reference[refs]{}%

\stopchapter
\stopcomponent